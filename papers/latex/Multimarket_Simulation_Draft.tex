\documentclass[12pt]{article}

\usepackage[english]{babel}
\usepackage[utf8]{inputenc}
\usepackage{fancyhdr}
\usepackage[margin=1in]{geometry}
\usepackage{amsmath}
\usepackage{amsfonts}
\usepackage{amssymb}
\usepackage{scrextend}
\usepackage{enumitem}
\usepackage{tikz}
\usepackage{pgfplots}
\usepackage{multirow,array}
\usepackage{float}
\usepackage{graphicx}


\begin{document}

%\vspace*{\stretch{1.0}}
\begin{center}
 	\Large\textbf{(Draft) Multi-market simulation}\\
 	\large\textit{Saketh Aleti}
\end{center}
\rule{\textwidth}{1pt}

	
\section{Market Model}	

\subsection{Definitions}

Suppose we have a system of supply and demand curves for $n$ commodities; each curve is a function of the prices of the other commodities. So, we have:
\begin{subequations}
\begin{align}
\textbf{Supply: } & Q_{s,i} = \alpha_{s,i} + \beta_{s,i,1} P_1 + ... + \beta_{s,i,n} P_n				\\
\textbf{Demand: } & Q_{d,i} = \alpha_{d,i} + \beta_{d,i,1} P_1 + ... + \beta_{d,i,n} P_n	
\end{align}
\end{subequations}

Now, suppose that this system is in equilibrium at time $t = 0$ where $P_0$ and $Q_0$ are vectors representing the equilibrium price and quantity. We then introduce a predefined shock, ${\alpha_{shock,i}} $, to each supply curve and define $ \alpha_{s,i}' = {\alpha_{s,i}}+ {\alpha_{shock,i}} $. Thus, the new supply curve is: 
\begin{equation}
\textbf{New Supply: }  Q_{s,i} = \alpha_{s,i}' + \beta_{s,i,1} P_1 + ... + \beta_{s,i,n} P_n  \tag{1c}
\end{equation}

Suppose this shock occurs at time $t=1$; at this moment, the price and quantity are not in equilibrium. We find the equilibrium price and quantity by solving:
$$ \alpha_{s,i}' + \beta_{s,i,1} P_1 + ... + \beta_{s,i,n} P_n = \alpha_{d,i} + \beta_{d,i,1} P_1 + ... + \beta_{d,i,n} P_n$$
Now, we use the following definitions for simplicity:
$$ 
\alpha=\begin{bmatrix}
 	\alpha_{s,1}' - \alpha_{d,1} \\
 	\vdots\\
 	\alpha_{s,n}' - \alpha_{d,n}
\end{bmatrix} 
,\quad
\beta=\begin{bmatrix}
\beta_{d,1,1} - \beta_{s,1,1} & \dots & \beta_{d,1,n} - \beta_{s,1,n} \\
\vdots & \ddots & \vdots \\
\beta_{d,n,1} - \beta_{s,n,1} & \dots & \beta_{d,n,n} - \beta_{s,n,n}
\end{bmatrix} 
,\quad
P=\begin{bmatrix}
P_1 \\
\vdots\\
P_n
\end{bmatrix} 
,\quad
Q=\begin{bmatrix}
Q_1 \\
\vdots\\
Q_n
\end{bmatrix} 
$$
We can then simplify the former equation into $\beta P = \alpha$ and solve to give us the equilibrium price; the equilibrium quantity then follows from a substitution into equation (1b) or (1c). 

\subsection{Dynamics}

Suppose, however, that all markets did not simultaneously fall into equilibrium. That is, assume the producers for each commodity set prices simultaneously for their commodity's market but not for the entire market consisting of all commodities. Then, at each time step, we have:
$$Q_{t} = \alpha_d + \beta_d P_t$$
$$P_{t+1} = \beta_s^{-1} (Q_{t} - \alpha_s)$$
Here, we can see the consumers simultaneously arrive at purchasing $Q_t$ at time $t$ given price $P_t$. Then, producers react to the quantity purchased by changing prices to $P_{t+1}$. The steady-state for this problem gives us the equilibrium solution. 

To find the steady-state, we first transform this problem into a linear differential equation. So, we begin by computing the change in price at each time $t$:
\begin{align*}
{\partial P}/{\partial t} &= P_{t+1} - P_t 		     \\
&= \beta_s^{-1} (Q_{t} - \alpha_s) - P_t				 \\
&= \beta_s^{-1} \alpha  + \beta_s^{-1} \beta_d P_t - P_t \\
&= \beta_s^{-1} \alpha  + (\beta_s^{-1} \beta_d - I) P_t
\end{align*}

From this, it follows that the steady-state is achieved when the change in price with respect to time is 0; let the steady-state solution be denoted as $P^*$. Therefore, setting the equation equal to zero, we have:

$$(\beta_s^{-1} \beta_d - I) P^* = -\beta_s^{-1} \alpha$$

To visualize this, the following graph visualizes the movement of the prices of two commodities towards the equilibrium. In the background is the vector field for this diff eq, and $P_{t=0}^T = (1,2)$ which converges to the equilibrium $(0.875,1.810)$.

\begin{center}
\includegraphics[scale = 0.5]{figures/vector_field}
\end{center}

Note that this solution does not contain boundary conditions for the prices. So, let us also assume we must have $P_t >= 0$. Then, we know that the market will 

\end{document}