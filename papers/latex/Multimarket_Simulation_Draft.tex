\documentclass[12pt]{article}

\usepackage[english]{babel}
\usepackage[utf8]{inputenc}
\usepackage{fancyhdr}
\usepackage[margin=1in]{geometry}
\usepackage{amsmath}
\usepackage{amsfonts}
\usepackage{amssymb}
\usepackage{scrextend}
\usepackage{enumitem}
\usepackage{tikz}
\usepackage{pgfplots}
\usepackage{multirow,array}
\usepackage{float}
\usepackage{graphicx}


\begin{document}

%\vspace*{\stretch{1.0}}
\begin{center}
 	\Large\textbf{(Draft) Multi-market simulation}\\
 	\large\textit{Saketh Aleti}
\end{center}
\rule{\textwidth}{1pt}

	
\section{Market Model}	

\subsection{Definitions}

Suppose we have a system of supply and demand curves for $n$ commodities; each curve is a function of the prices of the other commodities. So, we have:
\begin{subequations}
\begin{align}
\textbf{Supply: } & Q_{s,i} = \alpha_{s,i} + \beta_{s,i,1} P_1 + ... + \beta_{s,i,n} P_n				\\
\textbf{Demand: } & Q_{d,i} = \alpha_{d,i} + \beta_{d,i,1} P_1 + ... + \beta_{d,i,n} P_n	
\end{align}
\end{subequations}

Now, suppose that this system is in equilibrium at time $t = 0$ where $P_0$ and $Q_0$ are vectors representing the equilibrium price and quantity. We then introduce a predefined shock, ${\alpha_{shock,i}} $, to each supply curve and define $ \alpha_{s,i}' = {\alpha_{s,i}}+ {\alpha_{shock,i}} $. Thus, the new supply curve is: 
\begin{equation}
\textbf{New Supply: }  Q_{s,i} = \alpha_{s,i}' + \beta_{s,i,1} P_1 + ... + \beta_{s,i,n} P_n  \tag{1c}
\end{equation}

Suppose this shock occurs at time $t=1$; at this moment, the price and quantity are not in equilibrium. We find the equilibrium price and quantity by solving:
$$ \alpha_{s,i}' + \beta_{s,i,1} P_1 + ... + \beta_{s,i,n} P_n = \alpha_{d,i} + \beta_{d,i,1} P_1 + ... + \beta_{d,i,n} P_n$$
Now, we use the following definitions for simplicity:
$$ 
\alpha=\begin{bmatrix}
 	\alpha_{s,1}' - \alpha_{d,1} \\
 	\vdots\\
 	\alpha_{s,n}' - \alpha_{d,n}
\end{bmatrix} 
,\quad
\beta=\begin{bmatrix}
\beta_{d,1,1} - \beta_{s,1,1} & \dots & \beta_{d,1,n} - \beta_{s,1,n} \\
\vdots & \ddots & \vdots \\
\beta_{d,n,1} - \beta_{s,n,1} & \dots & \beta_{d,n,n} - \beta_{s,n,n}
\end{bmatrix} 
,\quad
P=\begin{bmatrix}
P_1 \\
\vdots\\
P_n
\end{bmatrix} 
,\quad
Q=\begin{bmatrix}
Q_1 \\
\vdots\\
Q_n
\end{bmatrix} 
$$
We can then simplify the former equation into $\beta P = \alpha$ and solve to give us the equilibrium price; the equilibrium quantity then follows from a substitution into equation (1b) or (1c). 

\subsection{Dynamics}

Suppose, however, that all markets did not simultaneously fall into equilibrium. That is, assume the producers for each commodity set prices simultaneously for their commodity's market but not for the entire market consisting of all commodities. Then, after the first supply shock at time $t=1$, we initially solve the following system:
\begin{align*}
Q_{1} &= \alpha_s' + \beta_s P_1 \\
P_{1}& = \alpha_d  + diag(\beta_d) P_1
\end{align*}
Here, $diag(\beta_d)$ refers to a matrix consisting only of the diagonal elements of $B_d$. Solving this system then results in a new non-equillibrium price that has yet to take into account shifts in demand due to cross elasticities.


Then, at each time step, we have:
\begin{align*}
Q_{t} &= \alpha_d + \beta_d P_t \\
P_{t+1}& = \beta_s^{-1} (Q_{t} - \alpha_s') \\
Q_{t+1} &= \alpha_s' + \beta_s P_{t+1}
\end{align*}
Here, we can see the consumers simultaneously arrive at purchasing $Q_t$ at time $t$ given prices $P_t$. Then, producers react to the new demand curves by changing production to $Q_{t+1}$ and selling at $P_{t+1}$. If $B_s$ is not a diagonal matrix, this stage actually results in a shift in supply curves as farmers start producing different crops to react to the demand. The steady-state for this problem, $P_t \approx P_{t+1}$, gives us the equilibrium solution. \\

To find the steady-state, we first transform this problem into a linear differential equation. So, we begin by computing the change in price at each time $t$:
\begin{align*}
{\partial P}/{\partial t} &= P_{t+1} - P_t 		     \\
&= \beta_s^{-1} (Q_{t} - \alpha_s) - P_t				 \\
&= \beta_s^{-1} \alpha  + \beta_s^{-1} \beta_d P_t - P_t \\
&= \beta_s^{-1} \alpha  + (\beta_s^{-1} \beta_d - I) P_t
\end{align*}

From this, it follows that the steady-state is achieved when the change in price with respect to time is 0; let the steady-state solution be denoted as $P^*$. Therefore, setting the equation equal to zero, we have:

$$(\beta_s^{-1} \beta_d - I) P^* = -\beta_s^{-1} \alpha$$

To visualize this, the following graph shows the movement of the prices of two commodities towards the equilibrium. In the background is the vector field for this diff eq. The initial point is $P_{t=0}^T = (1,2)$ which converges to the equilibrium $(0.875,1.810)$.

\begin{center}
	\includegraphics[scale = 0.5]{figures/vector_field}
\end{center}

However, the existence of an equilibrium point for this system does not necessarily imply that the prices will converge. For example, suppose we have the case where:

$$| P'_t | \geq | P'_{t-1} | $$

Here, $P'_t$ refers to the change in price at time $t$. This case is illustrated below, in a different system, where $P_0$ is marked by the red circle. Note that the vector field shows convergence, but the prices by increasingly larger amounts. 

\begin{center}
	\includegraphics[scale = 0.5]{figures/vector_field2}
\end{center}

We see that, rather than converging, prices are unstable actually diverge in this system. The equilibrium to the system, if all equations are instead solved simultaneously, is simply $(0.868,1.944)$. Yet, this equilibrium is only stable if it is reached precisely. If we take the system at equilibrium and shift prices by a very small amount, the system again diverges. We can see this in the example below; the initial price was set to be very close but not exactly equal to the equilibrium

\begin{center}
	\includegraphics[scale = 0.5]{figures/vector_field3}
\end{center}
. 

\end{document}